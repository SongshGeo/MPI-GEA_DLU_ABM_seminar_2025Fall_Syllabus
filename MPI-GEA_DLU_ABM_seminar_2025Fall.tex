\documentclass[a4paper,11pt]{article}
\usepackage[utf8]{inputenc}
\usepackage[english]{babel}
\usepackage[T1]{fontenc}

% 设置字体为 Times New Roman
\usepackage{times}
\usepackage{csquotes}

% 图形和表格支持
\usepackage{graphicx}
\usepackage{tabularx}
\usepackage{booktabs}
\usepackage{longtable}
\usepackage{array}

% CSV读取支持
\usepackage[l3]{csvsimple}

% 参考文献支持
\usepackage[
   style=authoryear,
   backend=biber,
   maxnames=20,
   giveninits=true,
   url=false,
   doi=false,
   isbn=false
]{biblatex}

% 设定页边距
\usepackage[margin=2.5cm]{geometry}

% 设定行距和间距
\usepackage{setspace}
\linespread{1.0}  % 减少行距

% 调整标题间距
\usepackage{titlesec}
\titlespacing*{\section}{0pt}{12pt plus 4pt minus 2pt}{8pt plus 2pt minus 1pt}
\titlespacing*{\subsection}{0pt}{10pt plus 4pt minus 2pt}{6pt plus 2pt minus 1pt}
\titlespacing*{\subsubsection}{0pt}{8pt plus 4pt minus 2pt}{4pt plus 2pt minus 1pt}

% 调整列表间距
\usepackage{enumitem}
\setlist{itemsep=2pt,parsep=2pt,topsep=4pt,partopsep=2pt}
\setlist[itemize]{leftmargin=1.5em,itemindent=0em,labelsep=0.5em}

% 调整段落间距
\setlength{\parskip}{3pt}
\setlength{\parindent}{0pt}

% 超链接支持
\usepackage{hyperref}
\hypersetup{
    colorlinks=true,
    linkcolor=blue,
    urlcolor=blue,
    citecolor=blue
}

% 微调排版
\usepackage{microtype}

% 加载配置文件
% Course Syllabus Minimal Configuration File
% 课程大纲简化配置文件

% =============================================================================
% BASIC COURSE INFORMATION
% 基本课程信息
% =============================================================================

% Course title and code
\newcommand{\CourseTitle}{Agent-Based Modeling for Urban and Archaeological Studies}
\newcommand{\CourseCode}{ABM-2025}
\newcommand{\CourseSemester}{Winter Semester 2025}

% Course credits and format
\newcommand{\CourseCredits}{3 ECTS}
\newcommand{\CourseFormat}{In-person and Virtual}
\newcommand{\CourseLocation}{DLU, Zoom}

% Course date
\newcommand{\CourseDate}{October 2025}

% Maximum number of schedule entries to display from CSV
\newcommand{\MaxScheduleEntries}{8}

% =============================================================================
% INSTITUTION AND LOGO SETTINGS
% 机构和Logo设置
% =============================================================================

% Institution information
\newcommand{\InstitutionName}{Max Planck Institute of Geoanthropology}
\newcommand{\InstitutionAddress}{Kahlaische Strasse 10, 07745 Jena, Germany}

% Logo settings
\newcommand{\LogoPath}{imgs/MPI-GEA_logo.pdf}
\newcommand{\LogoScale}{0.3}

% =============================================================================
% INSTRUCTOR INFORMATION - MULTIPLE INSTRUCTORS
% 教师信息 - 支持多个教师
% =============================================================================

% Primary instructor (main contact)
\newcommand{\AuthorName}{Song Shuang}
\newcommand{\AuthorFirstName}{Shuang}
\newcommand{\AuthorLastName}{Song}
\newcommand{\AuthorInstitution}{Max Planck Institute of Geoanthropology}
\newcommand{\AuthorAddress}{Kahlaische Strasse 10, 07745 Jena, Germany}
\newcommand{\AuthorEmail}{song@gea.mpg.de}
\newcommand{\AuthorPhone}{+491784979062}
\newcommand{\AuthorOrcid}{0000-0000-0000-0000}

% Course instructor (can be different from author)
\newcommand{\CourseInstructor}{Song Shuang}

% =============================================================================
% COURSE CONTENT - MINIMAL VERSION
% 课程内容 - 简化版本
% =============================================================================

% Course description
\newcommand{\CourseDescription}{%
Imagine sitting behind the wheel, inching forward on a crowded overpass. The brake lights ahead flicker on, one by one, and soon you find yourself slowing too. As the minutes pass, the stop-and-go rhythm ripples backward through the line of cars, spreading into a wave of congestion that sweeps across the city. From above, the pulse of red and white lights resembles a carefully rehearsed dance — a pattern no single driver intended, yet unfolding with the elegance of choreography. This course trains students to use computational agent-based models to explore such emergent dynamics, from traffic waves to the long-term coevolution of land use and urbanisation.
}

% Course participants
\newcommand{\CourseParticipants}{%
\item Postdocs and PhD students in the Department of Coevolution of Landuse and Urbanisation
}

% Course instructors (multiple instructors with proper formatting)
\newcommand{\CourseInstructors}{%
\item Chris Carleton (\href{mailto:carleton@gea.mpg.de}{carleton@gea.mpg.de})
\item Song Shuang (\href{mailto:song@gea.mpg.de}{song@gea.mpg.de})
}

% Course instructors list for title page (formatted for title page)
\newcommand{\CourseInstructorsList}{%
\textbf{Chris Carleton}\\
Max Planck Institute of Geoanthropology\\
Kahlaische Strasse 10, 07745 Jena, Germany\\
\textbf{Email:} \href{mailto:carleton@gea.mpg.de}{carleton@gea.mpg.de}\\[0.5cm]
\textbf{Song Shuang}\\
Max Planck Institute of Geoanthropology\\
Kahlaische Strasse 10, 07745 Jena, Germany\\
\textbf{Email:} \href{mailto:song@gea.mpg.de}{song@gea.mpg.de}\\
\textbf{Phone:} +491784979062
}

% German course equivalent
\newcommand{\CourseGermanEquivalent}{%
1 SWS, 10-13 hours of instruction
}

% Course format details
\newcommand{\CourseFormatDetails}{%
\item in person, DLU
\item virtual, Zoom
}

% =============================================================================
% BIBLIOGRAPHY SETTINGS
% 参考文献设置
% =============================================================================

% Note: Multiple bibliography files are now loaded directly in the main LaTeX file
% 注意:多个参考文献文件现在直接在主LaTeX文件中加载

% =============================================================================
% FORMATTING SETTINGS
% 格式设置
% =============================================================================

% Page layout
\setlength{\parindent}{0pt}
\setlength{\parskip}{6pt}

% Table formatting
\setlength{\arrayrulewidth}{0.5pt}
\renewcommand{\arraystretch}{1.2}


% 设置课程信息
\title{\CourseTitle}
\date{\CourseSemester}

% 添加参考文献资源 - 支持多个文件
% 直接加载多个BibTeX文件
\addbibresource{refer/lecture1.bib}
\addbibresource{course_bibliography.bib}
\addbibresource{refer/additional_refs.bib}

% 自定义命令
\newcommand{\sectiontitle}[1]{\section*{#1}}
\newcommand{\subsectiontitle}[1]{\subsection*{#1}}

% 防止孤行和寡行
\clubpenalty=10000
\widowpenalty=10000

\begin{document}

% 标题页
\begin{titlepage}
\centering

% 机构Logo
\ifx\LogoPath\empty\else
\vspace*{0.5cm}
\includegraphics[width=0.8\textwidth]{\LogoPath}
\vspace{1cm}
\fi

% 课程标题
{\Huge \textbf{\CourseTitle}}
\vspace{1cm}

% 课程信息
{\Large \CourseCode} \quad \textbf{\CourseSemester}
\vspace{0.5cm}

% 学分信息
\textbf{Credits:} \CourseCredits \quad 
\textbf{Format:} \CourseFormat \quad 
\textbf{Location:} \CourseLocation
\vspace{1cm}

% 教师信息 - 支持多个教师
% \subsectiontitle{Instructors}
\CourseInstructorsList

\vfill
\CourseDate
\end{titlepage}

% 课程描述
\sectiontitle{Course Description}
\CourseDescription

% 参与者
\sectiontitle{Participants}
\begin{itemize}
\CourseParticipants
\end{itemize}

% 教师信息
\sectiontitle{Instructors}
\begin{itemize}
\CourseInstructors
\end{itemize}

% 德语课程等价
\sectiontitle{German Course Equivalent}
\CourseGermanEquivalent

% 课程格式
\sectiontitle{Format}
\begin{itemize}
\CourseFormatDetails
\end{itemize}

% 课程安排
\sectiontitle{Schedule}

\subsectiontitle{Overview}

This seminar is planned to be held \MaxScheduleEntries times. The table below lists the schedule and content for each session. It will be adjusted flexibly as the course progresses.

% 从CSV文件读取课程安排
\begin{center}
\footnotesize
\setlength{\extrarowheight}{2pt}  % 减少表格行高
\begin{longtable}{p{2cm} p{2cm} p{5cm} p{5cm}}
\toprule
\textbf{Date} & \textbf{Time} & \textbf{Topic} & \textbf{Readings / Assignments} \\
\midrule
\csvreader[
  head to column names,
  late after line=\\,
  range={1-\MaxScheduleEntries}
]{course_schedule.csv}{
  date=\colA, time=\colB, topic=\colC, readings=\colD
}{
  \colA & \colB & \colC & \colD
}
\bottomrule
\end{longtable}
\end{center}

\newpage
\subsectiontitle{Lecture 1}

\begin{itemize}
\item \textbf{30 mins: Model thinking: Introduction of basic concepts of ABM. Shuang Song}
	\begin{itemize}
	\item Understanding the key terms \& components ($\sim$15min)
	\item Quick review of a classical model: segregation ($\sim$10min)
	\item Basic process \& references of ABM research ($\sim$5min)
	\end{itemize}
\item \textbf{30 mins: ABM with Archaeological Cases. Chris}
	\begin{itemize}
	\item Case 1
	\item Case 2
	\end{itemize}
\item \textbf{30 mins: Basic components of programming. Chris}
	\begin{itemize}
	\item Data Types: Variable, Container, Loop ...
	\item Function: Argument, Parameters, Returns...
	\end{itemize}
\end{itemize}

I recommend consulting some literature on the basic concepts of components related to agent-based models (ABMs) \cite{beckage2022, schulze2017, romanowska2019, matthews2007, schluter2023}. Here are some textbooks for beginners \cite{crooks2018, miller2022, wurzer2014}. Thomas Schelling's segregation model, developed in the 1970s, illustrates how subtle individual preferences for similarity can give rise to pronounced patterns of spatial segregation \cite{schelling1971}. You can find a runnable version with \href{http://nifty.stanford.edu/2014/mccown-schelling-model-segregation/}{a detailed explanation here}. Then, Lim et al. \cite{lim2007} applied the Schelling framework to study the relationship between racial segregation and violent conflict. Their research demonstrated how spatial separation can exacerbate intergroup tensions and contribute to cycles of violence in urban environments. Even nowadays, the Schelling model is still widely applied by other scientists. Recent advances by Seara et al. \cite{seara2025} extend the Schelling model to analyse social dynamics and residential preferences in depth.


% 推荐资源
\sectiontitle{Reference \& Additional Resources}

\subsectiontitle{Papers and Books}
% 临时调整参考文献列表间距
\setlength{\bibitemsep}{3pt}
\setlength{\bibparsep}{3pt}
\printbibliography[title={},heading=none]

\subsectiontitle{Websites, Wikis, Docs}

\begin{itemize}
\item Mesa: Agent Based Modelling in Python. \href{https://mesa.readthedocs.io/latest/}{https://mesa.readthedocs.io/latest/}
\item NetLogo \href{https://www.netlogo.org/}{https://www.netlogo.org/}
\item GIS Agents \href{https://www.gisagents.org/}{https://www.gisagents.org/}
\end{itemize}

\end{document}

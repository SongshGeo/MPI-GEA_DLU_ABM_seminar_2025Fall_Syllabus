\subsectiontitle{Lecture 1}

\begin{itemize}
\item \textbf{30 mins: Model thinking: Introduction of basic concepts of ABM. Shuang Song}
	\begin{itemize}
	\item Understanding the key terms \& components ($\sim$15min)
	\item Quick review of a classical model: segregation ($\sim$10min)
	\item Basic process \& references of ABM research ($\sim$5min)
	\end{itemize}
\item \textbf{30 mins: ABM with Archaeological Cases. Chris}
	\begin{itemize}
	\item Case 1
	\item Case 2
	\end{itemize}
\item \textbf{30 mins: Basic components of programming. Chris}
	\begin{itemize}
	\item Data Types: Variable, Container, Loop ...
	\item Function: Argument, Parameters, Returns...
	\end{itemize}
\end{itemize}

I recommend consulting some literature on the basic concepts of components related to agent-based models (ABMs) \cite{beckage2022, schulze2017, romanowska2019, matthews2007, schluter2023}. Here are some textbooks for beginners \cite{crooks2018, miller2022, wurzer2014}. Thomas Schelling's segregation model, developed in the 1970s, illustrates how subtle individual preferences for similarity can give rise to pronounced patterns of spatial segregation \cite{schelling1971}. You can find a runnable version with \href{http://nifty.stanford.edu/2014/mccown-schelling-model-segregation/}{a detailed explanation here}. Then, Lim et al. \cite{lim2007} applied the Schelling framework to study the relationship between racial segregation and violent conflict. Their research demonstrated how spatial separation can exacerbate intergroup tensions and contribute to cycles of violence in urban environments. Even nowadays, the Schelling model is still widely applied by other scientists. Recent advances by Seara et al. \cite{seara2025} extend the Schelling model to analyse social dynamics and residential preferences in depth.
